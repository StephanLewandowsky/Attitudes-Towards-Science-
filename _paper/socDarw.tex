%Send to all psychonomics DE contributors, Keri,

%Hyde05
%https://qz.com/1190996/scientific-research-shows-gender-is-not-just-a-social-construct/
%Fine13a
%https://institutions.newscientist.com/article/mg24132190-100-how-neuroscience-is-exploding-the-myth-of-male-and-female-brains/

\documentclass[fignum,man]{apa}\usepackage[]{graphicx}\usepackage[]{color}
%% maxwidth is the original width if it is less than linewidth
%% otherwise use linewidth (to make sure the graphics do not exceed the margin)
\makeatletter
\def\maxwidth{ %
  \ifdim\Gin@nat@width>\linewidth
    \linewidth
  \else
    \Gin@nat@width
  \fi
}
\makeatother

\definecolor{fgcolor}{rgb}{0.345, 0.345, 0.345}
\newcommand{\hlnum}[1]{\textcolor[rgb]{0.686,0.059,0.569}{#1}}%
\newcommand{\hlstr}[1]{\textcolor[rgb]{0.192,0.494,0.8}{#1}}%
\newcommand{\hlcom}[1]{\textcolor[rgb]{0.678,0.584,0.686}{\textit{#1}}}%
\newcommand{\hlopt}[1]{\textcolor[rgb]{0,0,0}{#1}}%
\newcommand{\hlstd}[1]{\textcolor[rgb]{0.345,0.345,0.345}{#1}}%
\newcommand{\hlkwa}[1]{\textcolor[rgb]{0.161,0.373,0.58}{\textbf{#1}}}%
\newcommand{\hlkwb}[1]{\textcolor[rgb]{0.69,0.353,0.396}{#1}}%
\newcommand{\hlkwc}[1]{\textcolor[rgb]{0.333,0.667,0.333}{#1}}%
\newcommand{\hlkwd}[1]{\textcolor[rgb]{0.737,0.353,0.396}{\textbf{#1}}}%
\let\hlipl\hlkwb

\usepackage{framed}
\makeatletter
\newenvironment{kframe}{%
 \def\at@end@of@kframe{}%
 \ifinner\ifhmode%
  \def\at@end@of@kframe{\end{minipage}}%
  \begin{minipage}{\columnwidth}%
 \fi\fi%
 \def\FrameCommand##1{\hskip\@totalleftmargin \hskip-\fboxsep
 \colorbox{shadecolor}{##1}\hskip-\fboxsep
     % There is no \\@totalrightmargin, so:
     \hskip-\linewidth \hskip-\@totalleftmargin \hskip\columnwidth}%
 \MakeFramed {\advance\hsize-\width
   \@totalleftmargin\z@ \linewidth\hsize
   \@setminipage}}%
 {\par\unskip\endMakeFramed%
 \at@end@of@kframe}
\makeatother

\definecolor{shadecolor}{rgb}{.97, .97, .97}
\definecolor{messagecolor}{rgb}{0, 0, 0}
\definecolor{warningcolor}{rgb}{1, 0, 1}
\definecolor{errorcolor}{rgb}{1, 0, 0}
\newenvironment{knitrout}{}{} % an empty environment to be redefined in TeX

\usepackage{alltt}

\usepackage{Sweave}
\usepackage{apacite}
\usepackage{longtable}
\usepackage{graphicx}
\usepackage{amsmath}
\usepackage{rotating}
\usepackage{multirow}
\usepackage{todonotes}

%\usepackage{draftwatermark}
%\SetWatermarkText{Please do not cite}
%\SetWatermarkScale{0.5}

\usepackage[hyphens]{url}
\usepackage{endfloat}
\widowpenalty=10000
\clubpenalty=10000
\raggedbottom
\newcommand{\helv}[1]{{\Huge\fontfamily{phv}\selectfont{#1}}}

\makeatletter
\def\input@path{{"../_outputTex/"}}
\makeatother
\graphicspath{{"../_figures/"}}

\title{Genesis or Evolution of Gender Differences? Worldview-based Dilemmas in The Processing of Scientific Information}
	
\threeauthors{Stephan Lewandowsky}{Jan K. Woike}{Klaus Oberauer}
%or \threeauthors{}{}{}
\threeaffiliations{University of Bristol and University of Western Australia}
{Max Planck Institute for Human Development}{University of Zurich}
\abstract{Some issues that have been settled by the scientific community, such as evolution, the effectiveness of vaccinations, and the role of CO$_2$ emissions in climate change, continue to be rejected by segments of the public. This rejection is typically driven by people's worldviews,
and to date most research has found that conservatives are uniformly more likely to reject scientific
findings than liberals across a number of domains.
We report a large (N$>$1,000) preregistered study that (a) examined people's attitudes
towards complementary and alternative medicines (CAM) and (b) examined 	how liberals and conservatives resolve dilemmas in which an issue triggers two opposing facets of their worldviews. We probed attitudes on gender equality and the evolution of sex differences---two constructs that may create conflicts for liberals (who endorse evolution but also equality) and conservatives (who endorse gender differences but are sceptical of evolution). We find that many conservatives reject both gender equality and evolution of sex differences, and instead embrace ``naturally occurring'' gender differences. Many liberals, by contrast, reject evolved gender differences, as well as naturally occurring gender differences, while nonetheless 
strongly endorsing evolution. We find more support for CAM among conservatives than
liberals.
}

\acknowledgements{Address correspondence to the first author at the 
School of Psychological Science,
University of Bristol,
12a Priory Road,
Bristol BS8 1TU, United Kingdom. email: stephan.lewandowsky@bristol.ac.uk.
Personal web page: http://www.cogsciwa.com.}

\shorttitle{Worldview-based Dilemmas}
\rightheader{Worldview-based Dilemmas}
\leftheader{Worldview-based Dilemmas}

\ifapamodeman{%
\note{Word count: 10,000 excluding references (approximate count due to use of \LaTeX)

\begin{flushleft}
Stephan Lewandowsky \\
School of Psychological Science and Cabot Institute \\
University of Bristol \\
12a Priory Road \\
Bristol BS8 1TU, United Kingdom \\
stephan.lewandowsky@bristol.ac.uk \\
URL: http://www.cogsciwa.com \\
\end{flushleft}}
}
{% else, i.e., in jou and doc mode
\note{}
}
\IfFileExists{upquote.sty}{\usepackage{upquote}}{}
\begin{document}
\maketitle


There is no scientific debate about the fact that all species, including humans, evolved by a process of natural selection \cite{PRC15}. There is no debate in the 
medical community about the
vast improvement to public health that has resulted from
widespread childhood vaccinations \cite{Whitney14}. 
There is also overwhelming evidence that many forms
of complementary and alternative ``medicine'' (CAM), such
as homeopathy, are ineffective. Reliance on CAM can 
even lead to unnecessary deaths if it 
causes cancer patients 
to refuse or delay evidence-based treatment \cite{Johnson18}.

%*** The Chan18 paper is only a review but it is quite detailed and gives us a good idea of what the current thinking is among the anti-liberal-bias brigade. The Cofnas article is also interesting because it blames scientists (who else) about conservatives’ opinions of science. Co-author Carl is a rightwing extremist.
%
%**** In the meantime, here are a few sobering telegrams from reality:
%\url{https://www.washingtonpost.com/news/wonk/wp/2015/06/02/this-astonishing-chart-shows-how-republicans-are-an-endangered-species/?utm_term=.ef1690d365f9 }
%
%\url{https://www.vox.com/policy-and-politics/2018/8/3/17644180/political-correctness-free-speech-liberal-data-georgetown}

The scientific consensus for evolution and vaccinations, and against complementary ``medicine'', stands in sharp contrast to the
opinions of a sizable segment of the public. 
For example, whereas 98\% of scientists accept that
humans evolved over time, only 65\% of the American
public shares that view \cite{PRC15}. Similarly, whereas 86\% of scientists believe that 
childhood vaccinations should be mandatory, this view is only 
shared by 68\% of the public \cite{PRC15}.
The presence of contrarian public opinions, even if they 
are only held by a minority, can have adverse 
consequences: anti-vaccination movements have had 
discernable impact
on public health \cite{Gangarosa98,Smith07b} and
organizations that oppose evolution have 
undermined science curricula in many American school 
districts \cite{Watts17}.

In consequence, there has been increasing research interest in
the variables that explain why people reject scientific facts. 
Two consistent findings have emerged from this
research:
First, educational attainment,
scientific knowledge, and science literacy
are at best modestly predictive of attitudes concerning scientific issues
\cite<e.g.,>{Allum08,Tom18}. 
Second, 
people's worldviews, that is their deeply-held beliefs 
about the world and how society should be 
organized, have been identified as the
preeminent predictor of attitudes towards scientific evidence
across numerous topics. 
In particular, in American participants, the rejection
of science is principally associated with 
rightwing or libertarian 
worldviews. Whether it is climate change \cite<e.g.,>{Hamilton11,Hamilton15b,Lewandowsky13b}, 
vaccinations \cite<e.g.,>{Hamilton15a,Kahan10a,Lewandowsky13b}, evolution \cite<e.g.,>{Hamilton15c,Tom18},
genetically-modified organisms \cite<e.g.,>{Hamilton15c}, or even
nuclear energy \cite<e.g.,>{Hamilton15c}, people on the political left 
trust scientists more on those issues
and tend to accept the pertinent scientific findings more than
their counterparts on the political right. 

Two important questions, however, remain unresolved: first, 
are there any domains in which
the role of worldviews is reversed---that is, do American 
liberals reject well-established scientific findings that conservatives endorse? 
Second, how do people respond to situations in which
their worldview provides conflicting imperatives that are
not readily reconcilable?
The present study was designed to address these two questions.

\subsection{Attitudinal symmetry and scientific evidence}
In many cases, the association
between rightwing worldviews and rejection of scientific 
evidence is easy to understand. For example, 
climate change is
a direct consequence of fossil-fuel powered economic growth, and 
successful
climate mitigation
will require cuts to greenhouse gas emissions \cite<e.g.,>{Knutti15}
that 
are not achievable without massive restructuring of the global
economy and large-scale deployment of new 
technologies \cite{Anderson16}. Accepting the existence and
origins of climate change 
is therefore tantamount to accepting that unregulated 
markets can create problems whose solution requires state intervention---clearly a challenging proposition for many conservatives
and libertarians. 
Similarly, libertarians may oppose public-health measures, such as
 mandatory 
childhood vaccinations, because they consitute
government intervention \cite{Kahan10a}. Evolution is typically
opposed for religious reasons \cite{Mazur04,Miller06c,Tom18}, and 
via the association between religiosity and 
rightwing worldviews \cite{Malka12}, this opposition will
also express itself when worldviews are measured to
predict attitudes towards evolution.

If conservatives' rejection of science arises because 
the evidence
challenges their political worldviews, then liberals might
likewise adopt problematic attitudes or reasoning strategies
when scientific evidence runs counter
to their own worldviews. Previous
attempts to discover scientific propositions that are
rejected by the political left 
have focused on genetically-modified organisms (GMO)
and vaccinations, based largely on anecdotal media
reports 
that claimed left-wing opposition to GMO foods \cite<e.g.,>{Shermer13W} 
and vaccinations \cite<e.g.,>{Mooney11W}. Those suggestions
have not withstood scrutiny 
\cite{Hamilton15c,Lewandowsky13b,Rabinowitz16}.\footnote{ Here we are
concerned only with attitudes towards scientific issues,
as revealed in surveys. We are not concerned with
laboratory experiments involving synthetic stimuli, which
have often shown ideological symmetry in people's
reliance on cognitive shortcuts \cite<e.g.,>{Washburn17}.} 

Here we continue our search for science denial on the left
by examining attitudes towards 
 complementary and alternative
medicine (CAM). 
CAM is 
particularly suitable for this search because
of anecdotal claims that 
alternative medicine 
and vaguely left-wing ideas may have found a symbiotic home
under the ``New Age'' umbrella \cite<see, e.g.,>{Keshet09}.  
Sociologists have also linked CAM use to ``resistance'' movements, such
as antipharmaceutical
activism or community development \cite{Gale14}. 
Homeopathy, for example, has been cited as ``feminist medicine'' 
\cite{Scott98a}. 
On the basis of this largely qualitative research one
might expect people on the political left
to be more hesitant to reject CAM
 despite the lack of scientific evidence supporting it
 than people on the right. 
%The 
%second reason we chose CAM is because 
%to date most examinations of science denial have involved 
%constructs whose \textit{endorsement} represented acceptance
%of scientific evidence. It is therefore conceivable
%that people on the political left are generally more disposed
%towards endorsing constructs. Most 
%extant research has sought to control an endorsement
%bias by using items of reversed polarity (i.e., agreeing
%with the item that ``climate change is a natural fluctuation''
%maps into rejection of mainstream climate science), but it is
%nonetheless possible that 

\subsection{Gender equality vs. differences}
One of the
core tenets of liberalism is the 
belief in the capacity to improve people
and their circumstances. 
This belief, often known as meliorism, 
is at the heart of liberalism \cite<e.g.,>{Castagno17,Porter13}, 
whereas disbelief or skepticism in that possibility
characterizes conservatives. 
Belief in the possibility of general human
improvement is therefore higher among liberals than
conservatives \cite{Miller99}. 
One long-standing and strong implication of liberal meliorism is 
belief in gender equality. 
Assuming that no important difference between the sexes is given by nature makes it easier to argue that all existing differences can be remedied by societal reform. 
Conservatives, by contrast, reject this possibility. 
A large body of literature
has found strong associations between conservatism and
sexism \cite<e.g.,>{Hodson17}, and between other indicators 
of rightwing politics such as Rightwing Authoritarianism (RWA) or 
social dominance orientation (SDO) and sexism \cite<e.g.,>{Hellmer18,VanAssche19}.
%VanAssche19: claims with citations that all religions are gender-unequal.

The scientific debate whether nature (i.e., biology, evolution, and
genetics) or nurture (i.e., social variables such as parenting and 
societal stereotypes) has a stronger influence on gender
differences has been raging for decades \cite<for a recent review, see>{Eagly13}. Arguably, some of the positions taken 
during this debate were shaped and constrained by ideology in addition
to data and evidence \cite{Eagly18}.
%Scarborough19: fewer traditionals, more egalitarians over time. Social survey across decades 
%Ellemers18: latest annual review of gender stereotypes
The involvement of ideology is unsurprising in light of
a long-standing and deep
fissure among American feminists and legal scholars between 
``sameness'' and 
``difference.'' Mid-twentieth-century feminism
laid claim to the essential equality between men and women by
highlighting their fundamental similarity. This 
emphasis on sameness shifted towards greater recognition of gender
differences in the closing decades of the twentieth century \cite{Williams89}. 
Both approaches share the goal of
achieving gender equality but they pursue quite different 
strategies. For example, when confronted with the issue of pregnancy
in the workplace, the approaches differ in 
``whether to stress the similarities between
men and women (in order to gain support for pregnant
women) or whether (and when) to stress their differences (in order
to gain support for pregnant women)'' \cite[p.~804]{Cain89}.
At one end of this continuum, scholars
search for ``feminist insights into women's true nature'' \cite[p.~4]{West88}. 
At the other extreme, scholars 
have replaced this ``essentialist'' view of women, 
whether biologically
or socially inspired, with radical social constructivism 
\cite<e.g.,>{MacKinnon89}. 

Although the scientific nature-nurture debate has not
been conclusively resolved \cite{Eagly13}, the available
evidence appears to rule out either extreme position. 
For example, a purely biological invariant essentialism is
challenged by the fact that gender-specific mate
preferences have changed considerably over the past 50 years.
Men
increasingly prefer women with
good financial prospects whereas housekeeping skills have
become less important, and conversely, 
women increasingly desire men with good looks
\cite<e.g.,>{Buss01}. Overall, there has been 
substantial convergence between the sexes in their stated
mate preferences during the past few decades.  
Similarly, a purely constructivist view
of gender differences is difficult to reconcile, at
first glance, with the fact that the more gender-equal
countries are, the \textit{greater} their gender gap in 
the number of graduates in science,
technology, engineering, and mathematics \cite{Stoet18}.
For example, 
Finland excels in gender equality but 
has one of the world's
largest gender gaps in science-based college degrees. 

For our study, the unresolved scientific
status of gender differences shifts
emphasis from comparing 
people's attitudes to a scientific ``gold standard''---as is possible
with issues such as evolution, vaccinations, or climate change--- 
to examining how people resolve dilemmas arising from 
conflicting imperatives of their worldview. 
It turns out that liberals' 
belief in gender equality gives 
rise to conflicting imperatives that are not necessarily easy to resolve.
Conservatives are similarly confronted with---different---gender-related dilemmas.
 
\subsection{Conflicting imperatives of worldview}
The fact that humans evolved is widely accepted
by people on the political left, and rejected by 
some on the right \cite{Tom18}.
Acceptance of evolution creates a potential
dilemma for liberals, given that some evolutionary
psychologists have been instrumental in drawing
scientific attention to ostensibly biologically-determined
gender differences, such as mate choice 
\cite<e.g.,>{Buss89}.\footnote{ Nothwithstanding the
ostensibly evolutionary constraints on mate choice illustrated
by \citeA{Buss89}, subsequent research
has shown that the preferences (e.g., for men who ``provide'' for women) 
are actually related to nations' gender parity \cite{Zentner12}.}
``To an evolutionary psychologist, the likelihood that
the sexes are psychologically identical in 
domains in which they have recurrently confronted different 
adaptive problems over the long expanse of human evolutionary history is essentially zero'' \cite[p.~301]{Buss96}.
How, then, will liberals reconcile their acceptance of Darwinian
evolution
with its potential detrimental impact, by some interpretations,
on another cherished aspect of
liberalism, namely meliorism and its tacit acceptance of gender
equality?

We are not aware of any research that has examined this
question in the public at large. However, 
some scholarly attention has focused on how members of 
scientific disciplines often identified
with a liberal orientation---namely, social psychology
and sociology---navigate the waters between Darwinian 
evolution and the implications of a variant of evolutionary 
psychology that postulates evolved differences in behavior and its
neural substrate.
\citeA{VonHippel17} asked a sample of more than 300 social psychologists 
about evolution and gender differences 
(and other traits and behaviors
not relevant here). The results showed that
the sampled scientists overwhelmingly accepted the theory of evolution,
providing a mean rating of 88\% on a scale from 0-100\% that Darwin's
ideas are likely to be true. By contrast, the mean ratings 
%But only 55\% agree that evolution also applies to ``human mind and many of our social attitudes and preferences evolved over millions of years''.
for the propositions that ``women's brains evolved to be more verbally talented'' and that ``men's brains evolved to be more mathematically talented'' 
were at 40\% and 30\%, respectively.
In another survey of sociologists, \citeA{Horowitz14} found that 43\% of
respondents found it plausible or highly plausible that ``differences between women and men in such skills as
communication and spatial reasoning are linked to biological
differences in female and male brains''. A further 22\% were undecided and only 35\% found this possibility implausible. 
Moreover, \citeauthor{Horowitz14} found that
self-identified feminist theoreticians were less likely to endorse
biological-evolutionary factors as underpinning social behavior
than sociologists with a different theoretical orientation.

These data suggest that although scientists generally accept
the importance of nature in explaining gender differences, they are
more skeptical of claims linking evolved differences
in brain structure to gender differences in behavior.
This skepticism appears amply justified by 
recent work in neuroscience
that has emphasized the plasticity of human brains 
as well as the fluidity of 
gender differences \cite<e.g.,>{Fine13,Fine13a,Fine14}.
Recent research on neuroplasticity may thus 
point to a resolution of the dilemma for scientists, 
but it remains to be seen how members of the public respond to the same dilemma.

The reverse dilemma confronts conservatives:
their known preference to reject gender equality could be 
buttressed by appealing to evolved, biological gender differences.
However, any such appeal 
would require at least tacit acceptance of Darwinian evolution, which 
would also be conflicting for many conservatives. How will
conservatives reconcile their reluctance to embrace evolution
with its potential utility in buttressing another cherished
aspect of conservatism, namely its endorsement of immutable 
gender differences?

Our study explored these worldview-based dilemmas for
liberals and conservatives by measuring 
three different constructs relating to gender differences.
We measured
people's beliefs about men and women being the same in all
respects, men and women having evolved differently,
and men and women being ``naturally different.'' 
The latter two constructs both allowed for an endorsement
of gender differences, but in one case those differences
were presumed to exist ``naturally'' without appealing
to evolution (or any other underlying causal process) 
whereas in the other case those differences 
were presented as the result of evolution. 
Figure~\ref{fig:dilemmas} illustrates the relationship
between our gender constructs, and participants' presumed
core beliefs. For liberals, we expected endorsement of
Darwinian evolution and gender equality to constitute core beliefs. Conversely,
for conservatives, rejection of evolution and rejection of 
gender equality were expected to constitute core beliefs.
The conflicts indicated in the figure follow from those core
beliefs: although evolved gender differences would explain
why men and women are different, this would be in conflict with
conservatives' rejection of evolution. Conversely,
for liberals the idea of evolved differences fits
well with acceptance of evolution but is in conflict with 
gender equality.
\begin{figure}[tp] %fig:dilemmas
	\fitfigure{dilemmas.pdf}
	\caption{Overview of the dilemmas facing liberals (left panel) and conservatives
	(right panel) in the context of gender differences and evolution. Core beliefs
    are shown in white (liberals) and black (conservatives) to reflect their
	opposing polarity. Potential explanatory variables for gender
	differences are shown in gray. Arrows represent presumed conflict between constructs,
	mutual support, or neutrality. }
	\label{fig:dilemmas}
\end{figure}

At least two possible resolutions of those dilemmas can be anticipated:
First, people might recruit the explanatory construct (e.g., evolved
gender differences) to support their core attitudes about gender. Thus,
conservatives might accept evolved gender differences because they buttress
their belief in differences between men and women, and liberals might
reject evolved gender differences because no such differences are presumed to exist.
This resolution would create conflict for both groups regarding their core attitudes
towards evolution. The second resolution would involve the reverse: both groups 
ensure that their attitudes involving evolution are internally consistent, 
with liberals endorsing
evolved gender differences and conservatives rejecting them. This resolution 
would create conflict for both groups regarding their core attitudes towards 
gender differences instead.

\section{Method}

\subsection{Overview}
We conducted a representative survey of 1,000 American residents.
The survey 
measured 10 constructs that belonged
to 5 conceptual groups:
(1) We measured people's endorsement of
two scientific constructs, namely evolution and vaccinations.
(2) We also measured people's attitudes towards
one instance of pseudoscience, namely
complementary and alternative medicines (CAM). 
To faciliate comparison with the other scientific constructs,
we coded this construct to represent \textit{rejection }of CAM
(and hence acceptance of the scientific view that
CAM is ineffective).
(3) We measured people's worldviews via three
related but distinct constructs; namely,
religiosity, conservatism, and endorsement of free markets.
(4) The three constructs related to gender differences (Figure~\ref{fig:dilemmas}).
(5) The final construct comprised the
three questions from the cognitive reflection test 
(CRT; \citeNP{Frederick05}).
The CRT measures people's propensity to engage in analytical reasoning.
In the present context, it is notable that CRT performance 
has been found to predict the ability to differentiate between ``fake news''
and accurate information, largely irrespective 
of whether or not the information
is worldview consonant \cite{Pennycook18}. 
Poor cognitive reflection may therefore also be associated with
denial of science. 
It is resonable to assume that cognitive reflection helps a person to 
evaluate the evidence for a proposition soberly, thereby 
increasing differentiation between science and pseudoscience or denial.

The sampling plan and procedure as well as an analysis
plan were preregistered before data collection
commenced. The preregistration document, including a complete
copy of the survey can be found on \textit{GitHub} at 
\url{https://git.io/fjpeB}.


\subsection{Materials}
The survey comprised 60 items, broken
down into 2 demographic queries presented at the outset (age and gender); 
14 items involving a ``slider'' scale (0-100) for the conservatism
construct; 
40 items on a 7-point scale (from ``Strongly Disagree'' to ``Strongly Agree'') 
to measure 
our remaining core attitudinal constructs;
the three items of the CRT;
and one item that served as attention filter. 
The attention filter asked people to identify which of a list of 5 items
was not an animal.

The conservatism items were taken from \citeA{Everett13} and asked participants to indicate ``the extent 
to which you feel positive or negative towards an issue'', with 50 taken
to be the neutral point and 0 representing great negativity and 
100 great positivity, respectively.
The 14 issues probed (and, where applicable, their short labels used
for presentation of the results) were
Abortion,
Welfare benefits (\textit{Welfare}),
Tax,
Immigration,
Limited government (\textit{LimGov}),
Military and national security (\textit{Military}),
Religion,
Gun ownership (\textit{Guns}),
Traditional marriage (\textit{TradMar}),
Traditional values (\textit{TradVal}),
Fiscal responsibility (\textit{FiscResp}),
Business,
The family unit (Family), and
Patriotism.
Each participant received the 14 slider scales 
in a uniquely-generated random order. 

The 40 items using a 7-point scale 
and the attention filter were presented in a 
different random order for each participant.
Table~\ref{tab:items} 
provides a verbatim list of these 40 items together
with brief labels (e.g., 
\emph{FMUnresBest} for ``An economic system based on free markets unrestrained by government interference automatically works best to meet human needs'') 
that are used for presentation of
the results.

The items for the three gender-related constructs were designed by
the authors for this study. (Pilot testing confirmed that their psychometric
properties were satisfactory.)
The items for the free-market and vaccination constructs 
were taken from our earlier research 
\cite<e.g.,>{Lewandowsky13b}. 
The items for the religiosity and evolution constructs 
were adapted from \citeA{Lombrozo08}.
The CAM-rejection construct
was probed by taking two items (\textit{CAMDanger} and \textit{CAMCure}) from \citeA{Hyland03}, and combining them with three other items developed by the authors.

\subsection{Ethics statement}
The Ethics Committee of the Max Planck Institute for Human Development
in Berlin approved the study. 
The survey was prefixed
by an introductory information sheet outlining the research.
Participants indicated their informed consent after reading this information sheet 
by 
a mouse click, which commenced presentation of the
survey questions.

\subsection{Participants and procedure}

A sample of 1,000 U.S. residents 18 years and older was recruited during June 2018
via electronic invitations by Qualtrics.com, a firm that
specializes in representative internet surveys. 
Participants were
drawn from a representative panel of more than 5.5 million
U.S. residents (as of January 2013), via propensity weighting to
ensure representativeness in terms of gender, age, and income.
Participants were compensated by Qualtrics using the company's
standard reward scheme.

As part of the sampling procedure, Qualtrics conducts a
``softlaunch'' ($N \simeq 50$) that provides an opportunity for
inspection of the data. We discovered that numerous participants 
in this preliminary sample 
responded identically (before reverse-coding)
to all items for one or more constructs. 
To deal with this indicator
of inattention, a further attention filter question was
inserted as part of the conservatism slider scale that asked
participants to select the value ``20''. Upon relaunch, 
an inspection of a further preliminary sample ($N=50$) suggested
that the additional attention filter
solved the inattention problem and hence sampling proceeded
for the full quota.

\section{Results}
The final sample, after exclusion of participants from 
the initial softlaunch, included 1017 responses that passed the Qualtrics
quality checks (including the two attention filters). The sample size
slightly exceeded the contracted quota of 1,000 because of a brief delay
between achieving the quota and shutting down of the survey.
This data file (stripped of geo-tagging and other potentially identifying 
information), 
the R scripts for all analyses, and the \LaTeX ~source file 
that weaves the results of the analysis directly into the paper can be found 
at \url{XXXXXX}. 

% received from Q: Attitudes_towards_science_evolution+raw from Q 19 June.csv
% downloaded and used for analysis: full1200downloadednotDeanom.csv

%Although our preregistered analysis plan did not specify any further exclusion criteria,
%we discovered that several participants responded identically (before reverse-coding)
%to all
%items for one or more constructs. We deemed this to be indicative of potential
%inattention and therefore eliminated any participant who responded identically 
%to all items for more than one construct.
%After exclusion, dim(relig15)[1]
%observations 
%were retained for analysis. 
The final sample included 489 men and 528 women, with a mean age of 
46.5 years
 (median 47;
range 18--99). 
Mean age differed between men (52.0) and women (41.4). 


\subsection{Data summary and adjustment}
Figure~\ref{fig:slidersSocDarw} shows the distribution
of slider responses to the 14 items of the conservatism scale.
Table~\ref{tab:itemResponses}
shows the number and percentages of responses to the 40 core
items before
reverse-scoring (for item labels, see Table~\ref{tab:items}).
The mean of the responses to all reverse-scored items (3.94)
was found to be 
closer to the midpoint of the scale (4) than the mean response to all non-reverse-scored
items (4.87), suggesting the presence of an
affirmation bias, that is a tendency to respond ``yes'' to any item regardless of
its polarity. For most constructs, this affirmation bias was at least partially controlled
by the inclusion of items of both polarities. However, the three gender-related sets 
of items did not include any reverse-scored items (with the entire ``men and women are the same''
cluster instead serving as reverse-polarity items for gender). We therefore
controlled the affirmation
bias for the gender-related items by statistical means.
We defined the mean response across all remaining items (i.e., all clusters other than
gender) before reverse scoring
as a person's affirmation-bias score. Responses to each of the gender items were regressed
on that affirmation-bias score, and the resulting residuals were added
to the mean response for that item to create adjusted responses. All
analyses are based on the affirmation-bias adjusted responses for the three 
gender constructs.
Because this affirmation bias was unexpected, the
adjustment could not be preregistered, and all remaining analyses
therefore depart from the preregistered analysis plan for the gender
constructs. The conclusions from this study are not materially altered if 
unadjusted responses are used instead. We do not report the 
unadjusted analysis in this article but all output and figures for the unadjusted
analysis can be found at \url{https://bit.ly/32nm3W9}.

Composite scores for each construct in Table~\ref{tab:itemResponses}
 were then formed by averaging 
responses across
all relevant items after reverse-scoring where necessary. 
Larger numbers refer to greater endorsement of a construct. 
Figure~\ref{fig:histoSocDarw} shows the distributions 
of the average scores for the 7 constructs. 
\begin{figure}[tp] %fig:slidersSocDarw
	\fitfigure{slidersSocDarw.pdf}
	\caption{Frequency distributions of responses for the
		14 items of the conservatism scale.
		Each histogram shows the distribution
		across subjects of the slider response which
		ranged from 0 (strong negativity) to 100 (high positivity). }
	\label{fig:slidersSocDarw}
\end{figure}

\begin{figure}[tp] %fig:histoSocDarw
	\fitfigure{histoSocDarw.pdf}
	\caption{Frequency distributions of the composite scores for all constructs
		excluding conservatism,
		formed by averaging responses across items within each construct 
		after
		reverse scoring. Each histogram shows the distribution
		across subjects of the composite score. 		
		Table~\ref{tab:items} provides an overview of the items for
		each construct. All items were accompanied by a 7-point 
		response scale ranging 
		from ``Strongly agree'' (coded as 7 for analysis) to
		 ``Strongly disagree'' (1) with ``Neither agree nor disagree'' (4)
		at the midpoint. }
	\label{fig:histoSocDarw}
\end{figure}

\subsection{Latent variable modeling}
Our
preregistered analysis plan identified structural equation modeling (SEM)
as our principal analysis technique. 
We therefore sought to represent each construct by a 
latent variable that was estimated from the 
responses to the corresponding set of items. 
Latent variables are free of measurement error, and thus none of the estimated effects 
are attenuated by measurement error \cite{Coffman05}. 
The analysis plan did not, however, 
specify particular SEM models, and our 
remaining analysis 
thus conformed to the analysis plan without being 
 prescribed by it.
 All SEM was conducted using the \textit{lavaan} package in R \cite{Rosseel12}.

SEM models with more than 20 items overall are often 
too complex and unwieldy to achieve adequate 
levels of model fit \cite{Bentler87}.
One way to overcome this problem 
is by averaging the item scores measuring 
each construct into a single-indicator 
variable for SEM, a procedure known as item parceling.
Averaging, however, may obscure multi-dimensionality \cite{Little02}. 
To retain the advantage of parceling without acquiring 
the problems arising from averaging, 
we first modeled each hypothesized 
latent variable using conventional SEM. These models
used each construct's respective items as a separate indicator 
variable and checked for the presence of multi-dimensionality.

\subsubsection{Measurement models}
To reduce potentially
problematic differences in variance between the slider variables (range 0--100)
and the remaining items, the slider results were rescaled to the range 1--7. 
All 8 constructs measured by the 7-point items 
exhibited an essentially uni-dimensional
structure, except that in all cases a 
correlation between the residuals of two items
had to be
added to the single-factor model to achieve a satisfactory fit.
Table~\ref{tab:indicatormodels} reports the fit statistics for  those 
8 measurement models. For the free-market and vaccination constructs that
had been employed in previous SEM modeling \cite{Lewandowsky13b}, 
the fit statistics were similar and the
correlated residuals involved the same items as before.
All models fit well or extremely well, with the possible 
exception of the measurement model for evolution, one of
whose fit indices was not satisfactory (RMSEA=$0.128$).

\begin{sidewaystable} %tab:indicatormodels
\caption{Model fit indices associated with the
	measurement models for all uni-dimensional constructs}
\label{tab:indicatormodels}
%\centering
\begin{tabular}{l rrrrrr l}
\thickline
\multicolumn{1}{c}{Construct}   & $\chi^2$ & $df$ & SRMR & CFI & RMSEA & \multicolumn{1}{c}{90\% CI} & \multicolumn{1}{c}{Correlated residuals} \\
\hline

Free market & 
10.24 & 
4 & 
0.019 & 
0.994 & 
0.039 &
0.009 -- 
0.07  & 
\textit{FMUnresBest}$\leftrightarrow$ \textit{FMMoreImp}
\\
Religiosity & 
41.93 & 
4 & 
0.026 & 
0.984 & 
0.097 &
0.071 -- 
0.124  & 
\textit{RelGod  }$\leftrightarrow$ \textit{RelAfterlife}
\\
 
Evolution & 
70.26 & 
4 & 
0.044 & 
0.952 & 
0.128 &
0.102 -- 
0.155  & 
\textit{EvoCreated  }$\leftrightarrow$ \textit{EvoCrisis}
\\

Vaccinations & 
14.21 & 
4 & 
0.017 & 
0.995 & 
0.05 &
0.024 -- 
0.079  & 
\textit{VaxNegSide   }$\leftrightarrow$ \textit{VaxRisky }
\\

Rejection of CAM & 
6.53 & 
4 & 
0.017 & 
0.996 & 
0.025 &
0 -- 
0.058  & 
\textit{CAMDanger   }$\leftrightarrow$ \textit{CAMIneffect }
\\

Men \& women evolved differently & 
3.46 & 
4 & 
0.012 & 
1 & 
0 &
0 -- 
0.044  & 
\textit{MWEvoViol   }$\leftrightarrow$ \textit{MWEvoNurture }
\\

Men \& women naturally different &
9.31 & 
4 & 
0.018 & 
0.993 & 
0.036 &
0 -- 
0.067  & 
\textit{MWNatAggress }$\leftrightarrow$ \textit{MWNatCaring }
\\

Men \& women are the same &
24.58 & 
4 & 
0.028 & 
0.972 & 
0.071 &
0.046 -- 
0.099  & 
\textit{MWEquDiff   }$\leftrightarrow$ \textit{MWEquInvent }
\\
Conservatism &
281.49 & 
33 & 
0.045 & 
0.939 & 
0.086 &
0.077 -- 
0.095  & 
\textit{TradMarriage   }$\leftrightarrow$ \textit{TradValues} \\
& & & & & & & 
\textit{Military   }$\leftrightarrow$ \textit{Patriotism} \\

\thickline
\end{tabular}
\end{sidewaystable}

Unlike for the other constructs, it proved impossible to create a unidimensional 
model for the 14 items of the conservatism scale. Inspection of the correlations among
items revealed that items of different polarity correlated little with each
other: that is, responses to the issues abortion, welfare, tax, and immigration
correlated little with the remaining issues, even after reverse scoring. 
We resolved this difficulty in two ways: First, we focused
exclusively on the 10 items
with a conservative polarity to create 
a uni-dimensional measurement model.
The last row in Table~\ref{tab:indicatormodels} contains the
fit statistic for this single-factor model, which 
had to be augmented with two pairwise correlations between residuals.
Second, 
  we created a composite score by averaging across all 
  14 items irrespective of polarity. 
  We report an analysis based on composites for all constructs (cf. Figure~\ref{fig:histoSocDarw}) 
  in parallel to the SEM modeling where appropriate.

\subsubsection{Single-indicator latent variable models}
Having confirmed the essentially unidimensional structure
of our contructs, we next
constructed single-indicator latent variables \cite{Hayduk96,Joreskog82}.
In single-indicator models, each latent variable is defined 
by one indicator consisting of an equally-weighted composite 
of the items (i.e., the mean score).
The true-score variance for each latent variable is then obtained by 
constraining the single-indicator's error variance 
to: $(1 - $reliability$) \times s^2$, 
where $s^2$ is equal to the composite score's total variance \cite{Joreskog82}.

An accurate 
estimator of reliability 
is $\omega$ \cite{Komaroff97,Raykov97}, which
  we estimated using the individual measurement 
models (Table~\ref{tab:indicatormodels}; 
for details, see \citeNP{Raykov97}).
The error variances of the indicators were set to 
the values shown in Table~\ref{tab:descriptives}
and all remaining SEM models used the single-indicator latent variables 
thus defined.
The present estimates of $\omega$ are in close agreement to the values observed by 
\citeA{Lewandowsky13b} for the constructs used by both studies (free market and vaccinations). 

\begin{table} %tab:descriptives
	\centering
	\caption{Summary statistics of single-indicator latent variable models}
	\label{tab:descriptives}

	\begin{tabular}{l r r c }
		\thickline
		\multicolumn{1}{c}{Construct}   & $s$ \tabfnm{\textit{a}}&  $\omega$ \tabfnm{\textit{b}}& $(1-\omega) \times s^2$ \tabfnm{\textit{c}} \\
		\hline
		Free market & 
		0.99 &
		0.58 &
		0.413 \\		
		
		Religiosity & 
		1.57 &
		0.85 &
		0.364 \\
		
		Evolution & 
		1.17 &
		0.69 &
		0.423 \\			
		
		Vaccinations & 
		1.28 &
		0.76 &
		0.389 \\		
		
        Rejection of CAM & 
		0.76 &
		0.4 &
		0.344 \\		
        
        
        Men \& women evolved differently & 
		0.94 &
		0.56 &
		0.394 \\		
        
        Men \& women naturally different & 
		1.01 &
		0.61 &
		0.394 \\		


		Men \& women are the same & 
		1.04 &
		0.63 &
		0.4 \\		
		
		Conservatism &
		1.2 &
		0.87 &
		0.19 \\		

		\thickline
	\end{tabular}
    \tabfnt{\textit{a}}{ Standard deviation of composite score}.
	\tabfnt{\textit{b}}{ $\sqrt{\omega}$
		corresponds to the loading of a 
		single-indicator manifest variable on its factor.} 
	\tabfnt{\textit{c}}{ Error variance of each single-indicator latent variable}
\end{table}


\subsubsection{Correlations among constructs}
Table~\ref{tab:lvcor} shows the correlation 
matrix for the single-indicator latent variables. 
All correlations
were significant at $p<.01$ or less, with the exception 
of those identified as ``\textit{ns}'' in the table.
The covariance matrix of latent variables
for this solution was not positive definite, likely reflecting
linear dependency between 
factors deriving from the high correlation between
the two constructs probing
differences between men and women (evolved differently vs.
naturally different). 

To ensure that the ill-conditioned covariance matrix did
not unduly alter the results,
Table~\ref{tab:compcor} shows the same correlations
based on the composite scores instead of latent variables.
As expected, those correlations are attenuated owing 
to measurement error compared to the correlations based
on latent variables. However, their pattern is identical
to that shown in Table~\ref{tab:lvcor}, suggesting 
that the latent-variable model is
adequately identified notwithstanding the ill-conditioned covariance
matrix.

The results replicate previous research, with a modest
negative correlation between free-market endorsement and acceptance of
vaccinations accompanied by a positive correlation between
conservatism and vaccinations. Both sign and 
magnitude of those correlations mesh well with previous research
\cite{Lewandowsky13b}. Likewise, the strong
negative correlation between religiosity and 
acceptance of evolution replicates existing research \cite<e.g.,>{Ecklund17},
as does the positive correlation between CAM rejection and
acceptance of vaccinations \cite<e.g.,>{Attwell18,Bryden18,Ernst02}.
The positive correlation between CAM rejection and evolution 
acceptance is in line with recent
reports that similar reasoning errors
underlie creationism and CAM acceptance \cite{WagnerEgger18}.

\begin{sidewaystable}[!htbp] \centering 
\caption{Correlations among latent variables} 
\label{tab:lvcor} 
\begin{tabular}{ l rr rr rr rr r} 
\\[-1.2ex]\hline 
\hline \\[-1.8ex] 
&
 \rotatebox[origin=c]{80}{Free market} & \rotatebox[origin=c]{80}{Evolution} &  \rotatebox[origin=c]{80}{Rejection of CAM} &  \rotatebox[origin=c]{80}{Men \& women evolved differently} &  \rotatebox[origin=c]{80}{Men \& women naturally different} & 
		 \rotatebox[origin=c]{80}{Men \& women are the same} & 
		 	\rotatebox[origin=c]{80}{Religiosity} & 
		 	 	 \rotatebox[origin=c]{80}{Vaccinations}  \\ 
\hline \\[-2ex]

Evolution & $-0.389$ & $$ & $$ & $$ & $$ & $$ & $$ & $$ & $$ & $$ \\ 
Rejection of CAM & $-0.182$ & $0.185$ & $$ & $$ & $$ & $$ & $$ & $$ & $$ & $$ \\ 
Men/women evolved differently & $0.179$ & $0.327$ & $0.048ns$ & $$ & $$ & $$ & $$ & $$ & $$ & $$ \\ 
Men/women naturally different & $0.426$ & $-0.307$ & $-0.020ns$ & $0.822$ & $$ & $$ & $$ & $$ & $$ & $$ \\ 
Men/women are the same & $-0.271$ & $0.331$ & $-0.009ns$ & $-0.350$ & $-0.669$ & $$ & $$ & $$ & $$ & $$ \\ 
Religiosity & $0.288$ & $-0.598$ & $-0.199$ & $-0.110$ & $0.256$ & $-0.246$ & $$ & $$ & $$ & $$ \\ 
Vaccinations & $-0.243$ & $0.352$ & $0.409$ & $-0.043ns$ & $-0.152$ & $0.192$ & $-0.062ns$ & $$ & $$ & $$ \\ 
Socio-pol conservatism & $0.484$ & $-0.315$ & $-0.199$ & $0.069ns$ & $0.400$ & $-0.351$ & $0.563$ & $-0.008ns$ & $$ & $$ \\ 
CRT & $-0.206$ & $0.269$ & $0.183$ & $-0.010ns$ & $-0.035ns$ & $-0.130$ & $-0.201$ & $0.197$ & $-0.009ns$ & $$ \\ 


\hline \\[-1.8ex] 
\end{tabular} 
\\
\textit{Note.} Correlations identified with ``$ns$'' are non-significant, $p>.10$. All others are significant at $p<.01$ or less.
\end{sidewaystable}


\begin{sidewaystable}[!htbp] \centering 
\caption{Correlations among composite measures for all constructs } 
\label{tab:compcor} 
\begin{tabular}{ l rr rr rr rr r} 
\\
\hline 
\hline \\
&
\multicolumn{1}{l}{
	\rotatebox[origin=c]{80}{Free market}} & \rotatebox[origin=c]{80}{Evolution} &  \rotatebox[origin=c]{80}{Rejection of CAM} &  \rotatebox[origin=c]{80}{Men/women evolved differently} &  \rotatebox[origin=c]{80}{Men/women naturally different} & 
\rotatebox[origin=c]{80}{Men/women are the same} & 
\rotatebox[origin=c]{80}{Religiosity} & 
\rotatebox[origin=c]{80}{Vaccinations} &
\rotatebox[origin=c]{80}{Conservatism} \\ 
\hline 
Evolution & $-0.251$ & $$ & $$ & $$ & $$ & $$ & $$ & $$ & $$ \\ 
Rejection of CAM & $-0.082$ & $0.086$ & $$ & $$ & $$ & $$ & $$ & $$ & $$ \\ 
Men/women evolved differently & $0.104$ & $0.208$ & $0.022ns$ & $$ & $$ & $$ & $$ & $$ & $$ \\ 
Men/women naturally different & $0.249$ & $-0.185$ & $-0.006ns$ & $0.496$ & $$ & $$ & $$ & $$ & $$ \\ 
Men/women are the same & $-0.166$ & $0.207$ & $-0.006ns$ & $-0.209$ & $-0.407$ & $$ & $$ & $$ & $$ \\ 
Religiosity & $0.198$ & $-0.454$ & $-0.119$ & $-0.087$ & $0.167$ & $-0.170$ & $$ & $$ & $$ \\ 
Vaccinations & $-0.159$ & $0.255$ & $0.220$ & $-0.033ns$ & $-0.108$ & $0.122$ & $-0.050ns$ & $$ & $$ \\ 
Socio-pol conservatism & $0.418$ & $-0.319$ & $-0.101$ & $0.091$ & $0.350$ & $-0.338$ & $0.520$ & $-0.053ns$ & $$ \\ 

\hline \\[-2ex] 
CRT & $-0.116$ & $0.167$ & $0.084$ & $0.006ns$ & $-0.008ns$ & $-0.077$ & $-0.144$ & $0.136$ & $-0.011ns$ \\ 
 
\hline
\end{tabular} 
\textit{Note.} Correlations identified with ``$ns$'' are non-significant, $p>.10$. All others are significant at $p<.01$ or less.
\end{sidewaystable}

Because three of our constructs related to gender
differences, we used participants' gender as a
grouping variable in two additional
SEM models of the correlations among latent variables
to examine whether men and women
differed in their attitude structures.
One model estimated all parameters independently for
men and women, and the second model constrained the covariances
among latent variables to be equal between groups.
The constrained model 
%was preferred based on AIC
%($AICs$ round(aovresult$AIC[1],3) vs. 
%round(aovresult$AIC[2],3) 
%for the full and constrained model, respectively), 
%although it 
incurred a significant loss of fit, 
$\chi^2$(36)$=$81.07, 
$p < 0.0001$, but 
because it
fit well by other measures, SRMR=
0.036; 
CFI=0.972; RMSEA= 
0.05, CI:
0.035 -- 
0.064,
we conclude that participants' gender did not substantially alter 
people's attitudinal structure. We therefore do not consider the
effects of participants' gender further.  

\subsubsection{Predictive model}

We next sought a unifying model to 
predict all scientific constructs, vaccinations, CAM rejection, and
evolution, simultaneously from the other predictors. 
The model was developed in three steps:
First, we identified individual predictive models for each of the 
scientific constructs on its own. In each case, we first 
fit a full model involving all potential predictor 
constructs (religiosity, conservatism, free-market endorsement, and the three
gender-related constructs)
and then eliminated predictors until we found the simplest
possible model that did not incur a significant loss of fit. 
Second, we fit a full model to predict all three scientific
constructs simultaneously, using the predictors for each
that were obtained in the first step.
Third, we constrained covariances
among predictors to zero until we found a simpler 
model that fit 
well and incurred no loss of fit compared to the full model ($\chi^2(2)
=2.62; p=0.27$). 
This final model is shown in Figure~\ref{fig:finalSEM} and
fit well, $\chi^2(10)=63.16$; 
SRMR=0.024; CFI=0.967; 
RMSEA=0.072 (90\% CI:
0.056--
0.09).
\begin{figure}[tp] %fig:finalSEM
\fitfigure{finalSEM.pdf}
\caption{Final predictive structural equation model. 
	All links and correlations shown are 
	standardized and significant; all $p \leq .05$.
	Indicator variables and their loadings, and disturbances on endogenous factors are not
	shown. Links between latent variables that are not shown are 
	constrained to zero.
	Loadings and variances of single-indicator latent 
	variables are reported in Tables~\ref{tab:indicatormodels}
	and~\ref{tab:descriptives}.}
\label{fig:finalSEM}
\end{figure}

The model captures the strong correlations (of varying polarity) among the gender-related constructs.
It also captures the correlations between the three scientific constructs (although 
unlike for the first-order correlations (Table~\ref{tab:lvcor}), 
CAM rejection
was not significantly correlated with evolution). 
This replicates other findings that people's attitudes towards scientific issues often covary \cite<e.g.>{Lewandowsky12a}. Accordingly, each of the two worldview constructs, conservatism
and free market, affected scientific issues in a uniform manner. Increasing free-market
endorsement predicted reduced acceptance of vaccination and evolution and reduced
rejection (i.e., greater endorsement) of CAM. Conversely, increasing conservatism
predicted an \textit{increase }in the acceptance of evolution and vaccination, although the 
effect on CAM rejection failed to reach significance. The role of conservatism in this model
differed from its first-order correlations with the scientific constructs. In particular, the negative first-order correlation between conservatism and evolution (Table~\ref{tab:lvcor}) here turned into
a positive regression coefficient. 
In light of the strong negative weight for religiosity, we interpret this to mean
that it is religiosity---not conservatism \textit{per se}---that is driving rejection
of evolution, with free-market endorsement being a second, weaker predictor. 


\subsubsection{Cognitive performance and attitudes}
Analysis of the three items for the CRT revealed that participants on average
achieved 0.47 correct (out of a possible 3).
This mean is lower than what is typically observed, although
mean and distribution of responses were commensurate with the lowest-performing
sample reported by \citeA{Frederick05}.
The percentage of participants who got 0, 1, 2, or 3 correct was 
70.7, 16.4, 8.3,
and 4.6, respectively. 
The bottom panel of Table~\ref{tab:compcor} shows the
correlations between CRT scores (formed by adding 1 
to a person's total for each correct response) 
and the other composite scores for all constructs.
The observed modest but significant negative correlation with
religiosity replicates previous results \cite{Gervais12a,Shenhav12,Stagnaro19}. 
\citeA{Jost17} reported a 
meta analysis of 13 studies that related CRT performance to political views.
The vast majority of those studies showed
that liberals exhibited more cognitive reflection than conservatives.
In the present data, this is echoed by the modest negative
correlation with free market, although it was not reflected in the
conservatism measure. 
The positive associations of the CRT with endorsement of all three
scientific constructs, vaccination, CAM rejection, and evolution
replicate a similar previous finding \cite{WagnerEgger18}.
The association also meshes well with recent findings that analytical thinking
is associated with better differentiation between ``fake news'' and
valid information \cite{Pennycook18}.  


\subsection{Nurture vs. nature vs. evolution}
This final analysis considered the relationship between people's
acceptance of evolution and the 
constructs that probed the origins and extent of gender differences. 
This analysis therefore explored the dilemmas postulated in 
Figure~\ref{fig:dilemmas}.
We divided participants according to their political views,
conducting a median split of the composite conservatism scale and additionally
considering only the top 25\% of conservative and liberal respondents.

Figure~\ref{fig:conslibevogrid} provides a first perspective on the results,
using composite scores for all constructs. 
The strong correlation between
the two constructs concerning the origin of gender differences (men and women evolved
differently vs. men and women are naturally different) is visible in all panels.
It is also clear that conservatives are more likely overall to think that
men and women differ naturally than liberals. Of particular
interest are the clusters of dark points, which represent people
who reject evolution. First, as expected from the pattern of correlations in 
Table~\ref{tab:compcor}, it is 
clear that there are more conservatives
than liberals who reject Darwinian evolution. Second, irrespective
of political orientation, it appears that people
who reject evolution strongly endorse gender differences---however,
notably that endorsement mainly involves the idea that men and women are
somehow ``naturally'' different, rather than the idea that they
``evolved differently.''
\begin{figure}[tp] %fig:conslibevogrid
\fitfigure{conslibevogrid.pdf}
\caption{Relationship between acceptance of evolution (represented by color
of plotting symbols) and people's responses to the constructs probing the
origin of gender differences. Composite scores are used for all constructs.
 Participants with liberal political
views are shown on the left and conservatives on the right. Top panels
represent the 25\% of participants who were most committed to their
political views and bottom panels represent a median split
along political views. Points are jittered to reduce over-printing.}
\label{fig:conslibevogrid}
\end{figure}

Another perspective on the results is provided by 
Figure~\ref{fig:conslibequalitygrid}
which again plots the two constructs concerning the origin of 
gender differences on the axes, but this time using color to represent the belief in 
gender equality. 
For liberals, belief in the three constructs is highly associated:
higher belief in equality (dark plotting symbols) is associated with low
belief in natural and evolved differences, and vice versa. 
For conservatives, by contrast, there is again a notable
cluster of people who do not believe that men and women evolved differently but
think they are naturally different. Those people, in the top left of
the right-hand panels, also show little belief in gender equality (light
plotting symbols).  

It is informative to compare liberals' responses across the two figures. Because
the axes are identical between figures, this comparison can identify the relationship
between people's belief in equality and evolution. Notably, the people in the bottom-left
quadrant who reject gender differences irrespective of their origin, are strong
believers in gender equality \textit{and }evolution generally. Conversely, the few
liberals who do not endorse gender equality (lightly colored plotting symbols in Figure~\ref{fig:conslibequalitygrid})
tend to be most skeptical of evolution (dark plotting symbols in 
Figure~\ref{fig:conslibevogrid}).
The same comparison for conservatives identifies more people who disbelieve evolution
and equality, and far fewer who endorse evolution generally but reject any
form of gender differences. 

\begin{figure}[tp] %fig:conslibequalitygrid
\fitfigure{conslibequalitygrid.pdf}
\caption{Relationship between the belief in gender equality (represented by
	color of plotting symbols) and people's responses to the constructs probing the
	origin of gender differences. 
	Composite scores are used for all constructs.
	Participants with liberal political
	views are shown on the left and conservatives on the right. Top panels
	represent the 25\% of participants who were most committed to their
	political views and bottom panels represent a median split
	along political views. Points are jittered to reduce over-printing.}
\label{fig:conslibequalitygrid}
\end{figure}

\section{Discussion}
\subsection{Relationship to previous results}
Our results coordinate well with multiple precedents in the literature,
which we take up for each of the
constructs examined.
Considering first religiosity, we replicate 
the substantial association between stronger religious beliefs
and conservatism \cite{Malka12,Schlenker12}. In our
study this association generalized across a broadly-defined
conservatism construct as well as a specific construct
targeting endorsement of laissez-faire free-market economics.
We also replicate the long-standing strong negative
association between religiosity and acceptance of evolution \cite<e.g.,>{Tom18} and the
modest negative association between religiosity and analytic thinking
(i.e., CRT performance) reported previously \cite{Jack16,Shenhav12,Stagnaro19}.
Likewise, the correlations between religiosity and the gender
constructs (e.g., Table~\ref{tab:lvcor}) are consistent 
with previous reports that religiosity predicts sexism \cite{VanAssche19},
although our results extend that previous finding because our
scales did not probe discriminatory sexism but the origin of presumed
gender differences.
The negative association between religiosity and CAM rejection 
is also unsurprising 
in light of previous research that has shown acceptance of CAM 
to be driven by intuitive thinking, paranormal beliefs,
and ontological confusions \cite{Lindeman11}.
At least one of those variables (intuitive thinking)
is also known to be a predictor of religiosity \cite<e.g.,>{Shenhav12}.

By contrast, our findings concerning religiosity also deviate from aspects
of other recent research
\cite{Rutjens17}. Unlike \citeauthor{Rutjens17}, we 
found no evidence of a link
between religiosity and rejection of vaccinations. 
Given that  \citeauthor{Rutjens17} observed this link
only in some of their studies and only for some measures
of religiosity (mainly measures of religious orthodoxy),
we are not concerned about this apparent 
departure from previous results. Indeed, in another
recent as-yet unpubslihed study involving identical constructs, 
we did observe
a negative association between vaccination and religiosity, 
suggesting that this relationship may well be real but is difficult to
observe consistently.

Attitudes towards vaccinations were instead
determined by the interplay of our two political constructs:
In our regression model, we found that 
conservatism was a positive predictor of
vaccination acceptance, whereas free-market
endorsement was a negative predictor. This 
nuanced interplay of the predictive role of two constructs that
are highly correlated ($r \simeq .5$)
replicates the pattern observed by \citeA{Lewandowsky13b}.
The different polarity of the effects of conservatism and 
free-market endorsement is 
consonant with the notion that libertarians object to the
government intrusion arising from mandatory vaccination programs
\cite{Kahan10a}, whereas liberals (i.e., people low in conservatism)
may oppose vaccinations 
because they distrust pharmaceutical companies \cite<e.g.,>{Attwell18}.
The strength of the latter relationship, however, was 
less than overwhelming, given that the first-order correlation
between conservatism and vaccination attitudes was non-significant
whereas the first-order negative correlation with free-market persisted
(Table~\ref{tab:lvcor}). 


\subsection{Rejection of science on the political left?}
Our findings provide 
little evidence that people on the 
political left reject vaccinations. Although
we found a positive link from conservatism to vaccination
attitudes in our predictive model (Figure~\ref{fig:finalSEM}),
the first-order correlation was non-significant (Table~\ref{tab:lvcor}).
Moreover, free-market endorsement, which is typically considered
a strand of conservatism \cite{Crowson09},
was a consistently strong negative predictor of vaccination attitudes,
whether as a first-order correlation or in the final predictive model.
Our results thus converge with other recent findings
that have found an association between right-wing politics and
rejection of vaccinations \cite{Baumgaertner18,Kahan10a,Lewandowsky13b,Rabinowitz16}.
In a recent cross-sectional analysis of voting behavior and vaccination rates across European countries, \citeA{Kennedy19} found a strong
relationship between the vote share for populist parties 
and vaccine hesitancy. 

Similarly, contrary to 
reports that CAM use 
and vaguely left-wing ideas 
have a natural affinity for each other \cite<see, e.g.,>{Keshet09},
we found that
CAM rejection was negatively, but modestly, associated with 
all three of our worldview constructs; namely, religiosity, 
free market endorsement, and conservatism (although
the link from conservatism was non-significant in the
predictive model; Figure~\ref{fig:finalSEM}). 
In our data, none of the gender
constructs were associated with CAM attitudes. This
runs counter to the idea that CAM use 
is ``feminist'' \cite{Scott98a}.
To our knowledge, our results constitute 
the first empirical examination of the links
between political
views and CAM attitudes. 
Our results that conservatives are more likely to embrace
CAM is consonant with 
historical analyses that have found strong links
between right-wing organizations, such as the
John Birch Society in the U.S., and endorsement
of ``alternative'' cancer treatments \cite{Markle78}.
The result also adds to the list of failed attempts
to discover science denial on the political left
\cite<e.g.,>{Hamilton11,Hamilton15a,Hamilton15b,Hamilton15c,Kahan10a,Lewandowsky13b,Tom18}. 

%Wilson18: efficacy of critical thinking teaching
\subsection{Attitudes towards gender differences}
We now consider the principal novel aspect of our study,
relating to the interplay of attitudes towards general Darwinian evolution,
gender differences, and how those gender differences
might have arisen.
The idea that men and women differ naturally was highly 
correlated with the idea that they evolved differently. 
Moreover, unsurprisingly, the  idea that men and women differ naturally 
was negatively correlated with
the construct that proclaimed their equality. 
The equality construct was 
also negatively
correlated with the idea that men and women evolved differently,
although that correlation was smaller than for 
natural differences. 

Intriguingly, overall acceptance of evolution was 
positively associated with two seemingly conflicting constructs; 
namely, that men and women evolved differently \textit{and} that they are
the same. Moreover, evolution was negatively correlated with the idea
that men and women are naturally different, even though evolution
is one way in which such ``natural'' differences might have emerged.
A similar divergent pattern obtained in the associations between
the constructs relating to gender differences and political 
attitudes. That is, whereas conservatism and free-market endorsement
were negatively correlated with gender equality, 
and strongly
positively correlated with men and women being naturally different, those
correlations were attenuated (for free market) or absent (for conservatism) 
for the idea that men and women evolved differently.
It thus appears that the involvement of evolution, either on its own or in explaining
gender differences, served as a ``wedge issue'' that disrupted
otherwise straightforward associations between right-wing politics
and opposition to gender equality (and, vice versa, rejection of
gender differences and left-wing politics) and---as foreshadowed
in Figure~\ref{fig:dilemmas}---created 
dilemmas for participants of all political persuasions.

Figures~\ref{fig:conslibevogrid} and~\ref{fig:conslibequalitygrid} illustrate
how participants resolve those dilemmas. 
Consider first conservatives (right-hand panels in both figures). 
Those who reject evolution tend to endorse ``natural'' gender differences
but reject evolved gender differences (Figure~\ref{fig:conslibevogrid}). These
individuals thus sit outside
the main cluster that captures the
otherwise strong association between the two gender-differences constructs.
Conservatives who are strongly committed to rejecting evolution
are thus willing to forego endorsement of gender differences if 
those are attributed to evolution. It is only when (nearly identical)
items appeal to ``natural'' differences that they are endorsed by 
those committed participants.
Conversely, liberals who are strongly committed to gender equality
tend to reject the idea of evolved gender differences, even
when those participants are committed to accepting evolution (Figure~\ref{fig:conslibequalitygrid}). 
Thus, partisans of either stripe can agree in their rejection of the idea that men and
women evolved differently, but they do so for entirely different
reasons. Conservatives do so when they are committed to reject evolution, and liberal do so when they are committed to gender equality. Both groups therefore resolve the dilemmas
posed by our gender constructs by ``sacrificing'' endorsement of evolved gender
differences.

\section{Conclusion}
Our results contribute to two seemingly conflicting streams 
of outcomes
in the literature on 
how worldviews moderate people's responses to scientific issues. 
One the one hand, there is much evidence for pervasive attitudinal
asymmetry, with conservatives being more likely to reject
well-established scientific propositions than liberals. To date,
little or no evidence for left-wing science denial has been reported.
We add to this stream by showing that, contrary to previous largely
anecdotal reports, liberals are more likely to reject complementary
and alternative medicines, in line with the scientific evidence, than
conservatives. 

On the other hand, there is considerable evidence that
liberals and conservatives \textit{process} scientific data
in a symmetrical fashion. That is, liberals and conservatives
alike resort to the same cognitive shortcuts when data conform to their
biases, giving rise to a symmetric set of errors \cite{Kahan17b,Washburn17}.
We also add to this stream of research by showing that, when confronted
by worldview-triggered dilemmas, both liberals and conservatives
resolve those dilemmas in an equally ``rational'' fashion,
by selectively ``sacrificing'' endorsement of a specific
construct about gender differences. 
Liberals who endorse evolution in general believe that for 
some reason it did not affect differences between the sexes; 
this could be rationalized perhaps by assuming that evolution causes 
differences only between but not within species. 
Conservatives who reject evolution believe that men and women differ 
naturally without having evolved differently; this could be 
rationalized by assuming, for instance, that those natural differences 
were the result of divine intervention.

%It is interesting to consider these data 
%within the context of recent research that has related feminist
%theory to evolution, and in particular
%the insights from evolutionary psychology. 
%There has been a long-standing tension, within science,
%between evolutionary psychologists, who purportedly study
%the way ``things are'', and feminist scholars who are alleged
%to be studying the way gender differences ``ought to be.'' 
%For example, \citeA{Buss11} ascribed to feminists various
%concerns, such as ``that if
%gender differences exist and are evolved, then some might
%claim that gender differences `ought' to exist'' (p.~770) or
%``that documentation of evolved differences might lead to
%justification of bad or immoral behavior'' (p.~770).
%On this view, feminists commit the naturalistic
%fallacy (confusing ``is'' with ``ought'') because the
%gender differences as described by 
%evolutionary psychology are ``facts'' that carry no normative
%implications. 
%
%In an analysis and rebuttal of those charges, \citeA{Nelson17} 
%proffered two arguments. First, 
%she underscores that legitimate normative (i.e., ``ought'') claims
%can be made on the basis of factual premises provided that 
%among those premises is a normative claim. For example, while 
%one cannot conclude that \textit{torture is wrong} from the factual premise that
%\textit{torture causes great suffering}, that same conclusion
%is permissible if the  same factual premise is accompanied by the ethical
%premise \textit{it is wrong to cause suffering}.  
%Because the factual claims of evolutionary psychology 
%are accompanied by tacit ethical premises embedded in their
%favourite metaphors---e.g., calling
%female eggs ``expensive'' and male sperm ``cheap''---feminist
%rebuttals of such factual claims do not commit the naturalistic
%fallacy even if they highlight equality. 
%In a nutshell, feminist scholars argue from premises 
%that they take to have 
%ethical content, and such premises permit ethical conclusions or
%counter-arguments.
%\citeA{Nelson17}'s second argument is 
%that many concepts contain both empirical and normative content.
%For example, Locke's notion that ``all men are equal'' does
%not entail the empirical claim that all men are equally intelligent
%or handsome or whatever. Rather, his claim subsumed the
%normative component that no one person
%should have rightful authority or privilege over another. By the same
%token when feminist scholars insist on gender ``equality'',
%they make the normative claim that no gender should be superior to another,
%without committing to the empirical hypothesis that there are no
%differences between men and women.
%Many of our liberal participants, by contrast, resolved the dilemma 
%by rejecting the existence of differences, however caused.
%
%***
%
%JW I found the ending a bit abrupt, here. The exchange between Buss and Nelson is not centered on the main question of the manuscript. Maybe it is possible to move to a more general level in a type of conclusion: For example, the study demonstrates that contradictory implications of endorsed positions are a promising avenue to study downstream inconsistencies, etc. Or a more general statement about the contribution could do the trick.  Chaning the order of reporting on CAM and gender differences would also help to end on a stronger note.
%
%***
%
%KO I find the last paragraph unsatisfying, for 2 reasons: (a) Nelson's 1st argument as portrayed here is weak: There is no logical path from value statements about the expensiveness of eggs to a normative statement about gender equality. 
%(b) Nelson's 2nd argument is sound, of course, but it is unclear what follows. Are we saying that feminist scholars can be (or are) consistent by endorsing evolved gender differences while endorsing gender equality in a normative (not descriptive) sense? That's true, of course, but it was not an option for our participants because we did not offer them a normative version of the gender-equality scale. So how is this relevant for our study? 


%\subsection{Need for consistency}
%
%--overarching value you are trying to protect, dilemma triggered by our questions.
%--revealed dilemma, possibly not triggered -- knowledge partitioning, hence no self-correction. Even experts partition. 

%\subsection{Alternative accounts}
%
%Bartz16; Reminders of Social Connection Can Attenuate Anthropomorphism
%
%Reject idea that ``social activism'' is responsible for reduction in trust in science among conservatives \cite{Cofnas18}.
%Not just our results but also experimental studies: \cite{Kotcher17}. Advocacy does not hurt.
%
%\subsection{Attitudinal asymmetry and the open question of cognitive asymmetry}
%On the one hand, a recent
%meta analysis has identified consistent cognitive differences
%on a number of measures
%between people on the political left and right \cite{Jost17}.
%The list of measures is extensive and includes
%assays of analytical reasoning such
%as the cognitive reflection test \cite{Frederick05}. 
%It also includes 
%experimental studies in which interventions
%were found to have different effects on people
%with different worldviews. For example, in one
%study conservatives 
%were found to be more likely to avoid seeking novel data in response
%to an open-ended question (e.g., ``justness'' of the world)
%than liberals \cite{Tullett16}. 
%
%On the other hand,  laboratory 
%experiments have repeatedly shown 
%that the propensity to engage in cognitive shortcuts---that is,
%responding on 
%the basis of superficial association rather than deep reflection---is distributed
%evenly across the political divide. 
%When participants are presented with synthetic data (e.g., hypothetical
%results of gun control laws) that are
%amenable to a quick---but inaccurate---interpretation,
%as well as to a competing complex---and accurate---interpretation, the
%quick but inaccurate interpretation is triggered when 
%it is worldview congruent, irrespective of the person's beliefs.
%Correspondingly, the more complex and accurate
%reading of the data is recruited only when the quick interpretation 
%challenges participants' worldview, and this effect also holds
%irrespective of a person's beliefs \cite{Kahan13a,Washburn17}. 
%These data suggest that when people are asked to engage in
%cognitive processing in the laboratory, as opposed to expressing
%their attitudes in a survey, then asymmetries
%across the political spectrum may not be observed.

%why self-identified liberals score higher on measures of scientific
%literacy \cite{Carl16}
%Sterling16 : individuals who endorsed neoliberal, free market ideology were more
%susceptible to “pseudo-profound bullshit,” that is, statements that were extremely vague or meaningless,
%such as: “Consciousness is the growth of coherence, and of us”; and “Your movement transforms
%universal observations.” Those who endorsed free market ideology scored lower on measures of verbal
%and fluid intelligence and were more receptive to “bullshit.”
% see also JARMAC paper
%On balance, 
%we do not consider the symmetry issue to be 
%fully resolved. 
%However, even if it ultimately turned out 
%that 
%people of all persuasions 
%respond in \emph{cognitively }symmetrical ways in the laboratory \cite<cf.>{Kahan13a,Washburn17},
%this cognitive symmetry would neither explain, nor be easily reconciled with,
%the pervasive \emph{attitudinal} asymmetry that is consistently observed in surveys.
%We therefore ask whether there is anything 
%about the scientific enterprise itself that may create challenges to people with
%rightwing or libertarian worldviews, and which may produce the attitudinal asymmetry.

\bibliography{mega,megaIPsub}

\pagebreak
\begin{longtable}{p{.2\linewidth} p{.8\linewidth}} %tab:items

	\caption[]{Items with 7-point response scale 
		used in the survey and their short names\label{tab:items}}\\
	\hline
	Item name & \multicolumn{1}{c}{Item (R = reverse scored)}  \\
	\hline

	\\
\hline

\multicolumn{2}{c}{1. Free market}\\
\nopagebreak
\hline
\nopagebreak
\emph{FMUnresBest}&An economic system based on free markets unrestrained by government interference automatically works best to meet human needs. \\
\emph{FMLimitSocial} &The free market system may be efficient for resource allocation  but it is limited in its capacity to promote social justice. (R)  \\
\emph{FMMoreImp}&The preservation of the free market system is more important than localized environmental concerns.  \\
\emph{FMThreatEnv} &Free and unregulated markets pose important threats to sustainable development. (R)  \\
\emph{FMUnsustain} &The free market system is likely to promote unsustainable consumption. (R) \\

	\\
	\hline
	\multicolumn{2}{c}{2. Religiosity}\\
	\nopagebreak
	\hline
	\nopagebreak
\emph{RelComf} & Do you agree with the following statement? "Religion gives me a great amount of comfort and security in my life."\\
\emph{RelGod} & I believe in God.\\
\emph{RelAfterlife} & I believe in some kind of afterlife.\\
\emph{RelNatWorld} & I do not think religion can or should make claims about the natural world. (R)\\
\emph{RelRelig} & I do not consider myself a religious person. (R)\\
	
	\\
\hline
\multicolumn{2}{c}{3. Evolution}\\
\nopagebreak
\hline
\nopagebreak
\emph{EvoAnimals} & I believe that animals have changed over time by a process of evolution.\\
\emph{EvoSupported} & I accept evolution by natural selection as a well-supported scientific theory.\\
\emph{EvoSpecies} & I believe that all species, including humans, have a common evolutionary origin.\\
\emph{EvoCreated} & I believe that species were created individually and do not change over time. (R)\\
\nopagebreak
\emph{EvoCrisis} & I believe that the theory of evolution by natural selection is in crisis and about to be overturned. (R)\\
	\\
	\hline
	
	\multicolumn{2}{c}{4. Vaccinations}\\
	\nopagebreak
	\hline
	\nopagebreak
	\emph{VaxSafe} & I believe that vaccines are a safe and reliable way to
	help avert the spread of preventable diseases. \\
	\emph{VaxNegSide} & I believe that vaccines have negative side effects that
	outweigh the benefits of vaccination for children. (R) \\
	\emph{VaxTested} & Vaccines are thoroughly tested in the laboratory and
	wouldn't be made available to the public unless it was
	known that they are safe. \\
	\emph{VaxRisky} &The risk of vaccinations to maim and kill children
	outweighs their health benefits. (R)  \\
	\emph{VaxContrib} & Vaccinations are one of the most significant
	contributions to public health. \\
	\hline	
	
	\\
\hline
\multicolumn{2}{c}{5. Rejection of Complementary and alternative medicine (CAM)}\\
\nopagebreak
\hline
\nopagebreak
\emph{CAMDanger} & Complementary medicine can be dangerous in that it may prevent people getting proper treatment.\\
\emph{CAMCure} & Complementary medicine builds up the body's own defenses, so leading to a permanent cure. (R)\\
\emph{CAMIneffect} & Homeopathy has been shown again and again to be ineffective as a cure for anything.\\
\emph{CAMSaves} & Complementary medicine has often saved the lives of patients who were already given up by conventional doctors. (R)\\
\emph{CAMSuperior} & Complementary medicine is superior to conventional medicine in treating chronic ailments such as allergies, headaches, and back pains. (R)\\

	\\
\hline
\multicolumn{2}{c}{6. Men and women evolved differently}\\
\nopagebreak
\hline
\nopagebreak
\emph{MWEvoDiff} & Men and women evolved to be different and these biological differences cannot be overcome by education.\\
\emph{MWEvoViol} & Evolutionary history has predisposed men more strongly than women towards violence.\\
\emph{MWEvoNurture} & Evolutionary history has predisposed women more strongly than men towards being helpful and nurturing.\\
\emph{MWEvoTraits} & All human traits are the product of evolution and therefore resist change.\\
\emph{MWEvoDiff2} & Thousands of years of evolution explain why differences between men and women are very difficult to overcome.\\

	\\
\hline
\multicolumn{2}{c}{7. Men and women are naturally different}\\
\nopagebreak
\hline
\nopagebreak
\emph{MWNatStrong} & Men are naturally stronger than women and those differences cannot be overcome by education.\\
\emph{MWNatAggress} & It is the in the nature of men to be physically aggressive more often than women.\\
\emph{MWNatCaring} & Women are naturally more caring and socially supportive than men.\\
\emph{MWNatTraits} & All human traits are part of our natural makeup and therefore very difficult to change.\\
\emph{MWNatDiff} & Men and women are naturally different from each other and those differences are bound to stay, even if we try hard to overcome them.\\

	\\
\hline
\multicolumn{2}{c}{8. Men and women are the same}\\
\nopagebreak
\hline
\nopagebreak
\emph{MWEqu} & Men and women are equally capable and powerful in all respects.\\
\emph{MWEquDiff} & All differences between men and women are created by society and can be eliminated if we change society.\\
\emph{MWEquCulture} & Without the pressures of culture and society women would be as much in control as men.\\
\emph{MWEquNoBio} & There are no biological or physical reasons that prevent a girl today to achieve as much as a boy.\\
\emph{MWEquInvent} & The categories ``male'' and ``female'' are primarily cultural inventions that have little basis in human nature.\\
\hline
	
\end{longtable}


\pagebreak
%\setlength{\tabcolsep}{8pt}
\begin{longtable}{p{.15\linewidth} rr rr rr rr rr rr rr} %tab:itemResponses
	\caption[]{Number of responses (percentages) for each response option for all survey items using a 7-point scale\label{tab:itemResponses}}\\
	\hline
	Item name & \multicolumn{2}{c}{\rotatebox[origin=c]{70}{Strongly disagree}} &
	\multicolumn{2}{c}{\rotatebox[origin=c]{70}{Disagree}} &
	\multicolumn{2}{c}{\rotatebox[origin=c]{70}{Somewhat disagree}} &
	\multicolumn{2}{c}{\rotatebox[origin=c]{70}{Neither agree nor disagree} } &
	\multicolumn{2}{c}{\rotatebox[origin=c]{70}{Somewhat agree}} &
	\multicolumn{2}{c}{\rotatebox[origin=c]{70}{Agree}} &
	\multicolumn{2}{c}{\rotatebox[origin=c]{70}{Strongly agree}}  \\
	
	\hline
	\multicolumn{15}{c}{1. Free market}\\
	\hline
	\nopagebreak
	FMUnresBest & 62 & (6) & 68 & (7) & 107 & (11) & 330 & (32) & 212 & (21) & 132 & (13) & 106 & (10)\\
FMLimitSocial  & 20 & (2) & 40 & (4) & 78 & (8) & 376 & (37) & 237 & (23) & 164 & (16) & 102 & (10)\\
FMMoreImp & 94 & (9) & 99 & (10) & 152 & (15) & 346 & (34) & 147 & (14) & 111 & (11) & 68 & (7)\\
FMThreatEnv  & 50 & (5) & 66 & (6) & 117 & (12) & 327 & (32) & 212 & (21) & 141 & (14) & 104 & (10)\\
FMUnsustain  & 49 & (5) & 78 & (8) & 114 & (11) & 379 & (37) & 179 & (18) & 134 & (13) & 84 & (8)\\


	\hline
\multicolumn{15}{c}{2. Religiosity}\\
\nopagebreak
\hline
\nopagebreak
\hline	
RelComf & 159 & (16) & 71 & (7) & 56 & (6) & 162 & (16) & 176 & (17) & 164 & (16) & 229 & (23)\\
RelGod & 87 & (9) & 30 & (3) & 29 & (3) & 111 & (11) & 84 & (8) & 150 & (15) & 526 & (52)\\
RelAfterlife & 59 & (6) & 34 & (3) & 28 & (3) & 146 & (14) & 122 & (12) & 231 & (23) & 397 & (39)\\
RelNatWorld & 129 & (13) & 98 & (10) & 90 & (9) & 359 & (35) & 110 & (11) & 88 & (9) & 143 & (14)\\
RelRelig & 228 & (22) & 161 & (16) & 112 & (11) & 115 & (11) & 108 & (11) & 129 & (13) & 164 & (16)\\
		
\hline

	
	\hline
	\multicolumn{15}{c}{3. Evolution}\\
	\nopagebreak
	\hline
	\nopagebreak
	EvoAnimals & 56 & (6) & 29 & (3) & 45 & (4) & 137 & (13) & 214 & (21) & 280 & (28) & 256 & (25)\\
EvoSupported & 90 & (9) & 46 & (5) & 55 & (5) & 240 & (24) & 210 & (21) & 193 & (19) & 183 & (18)\\
EvoSpecies & 80 & (8) & 43 & (4) & 54 & (5) & 206 & (20) & 205 & (20) & 240 & (24) & 189 & (19)\\
EvoCreated & 168 & (17) & 168 & (17) & 185 & (18) & 182 & (18) & 110 & (11) & 116 & (11) & 88 & (9)\\
EvoCrisis & 104 & (10) & 94 & (9) & 123 & (12) & 382 & (38) & 152 & (15) & 97 & (10) & 65 & (6)\\

		
	\pagebreak
	\hline
	Item name & \multicolumn{2}{c}{\rotatebox[origin=c]{70}{Strongly disagree}} &
	\multicolumn{2}{c}{\rotatebox[origin=c]{70}{Disagree}} &
	\multicolumn{2}{c}{\rotatebox[origin=c]{70}{Somewhat disagree}} &
	\multicolumn{2}{c}{\rotatebox[origin=c]{70}{Neither agree nor disagree} } &
	\multicolumn{2}{c}{\rotatebox[origin=c]{70}{Somewhat agree}} &
	\multicolumn{2}{c}{\rotatebox[origin=c]{70}{Agree}} &
	\multicolumn{2}{c}{\rotatebox[origin=c]{70}{Strongly agree}}  \\

	\hline
\multicolumn{15}{c}{4. Vaccinations}\\
\nopagebreak
\hline
\nopagebreak
\hline	
VaxSafe   & 41 & (4) & 16 & (2) & 44 & (4) & 111 & (11) & 157 & (15) & 252 & (25) & 396 & (39)\\
VaxNegSide   & 286 & (28) & 176 & (17) & 120 & (12) & 187 & (18) & 115 & (11) & 59 & (6) & 74 & (7)\\
VaxTested   & 44 & (4) & 31 & (3) & 77 & (8) & 183 & (18) & 217 & (21) & 268 & (26) & 197 & (19)\\
VaxRisky  & 262 & (26) & 171 & (17) & 116 & (11) & 265 & (26) & 72 & (7) & 63 & (6) & 68 & (7)\\
VaxContrib  & 27 & (3) & 10 & (1) & 40 & (4) & 158 & (16) & 195 & (19) & 285 & (28) & 302 & (30)\\
		
\hline

	\hline
	\multicolumn{15}{c}{5. Complementary and alternative medicine (CAM) }\\
	\hline
	\nopagebreak
	CAMDanger & 33 & (3) & 41 & (4) & 101 & (10) & 332 & (33) & 249 & (24) & 166 & (16) & 95 & (9)\\
CAMCure & 46 & (5) & 61 & (6) & 129 & (13) & 434 & (43) & 188 & (18) & 94 & (9) & 65 & (6)\\
CAMIneffect & 90 & (9) & 109 & (11) & 159 & (16) & 381 & (37) & 136 & (13) & 77 & (8) & 65 & (6)\\
CAMSaves & 18 & (2) & 17 & (2) & 41 & (4) & 381 & (37) & 264 & (26) & 183 & (18) & 113 & (11)\\
CAMSuperior & 46 & (5) & 64 & (6) & 103 & (10) & 443 & (44) & 177 & (17) & 111 & (11) & 73 & (7)\\
		

	\hline
	\multicolumn{15}{c}{6. Men and women evolved differently}\\
	\nopagebreak
	\hline
	\nopagebreak
   	MWEvoDiff & 104 & (10) & 103 & (10) & 144 & (14) & 208 & (20) & 176 & (17) & 159 & (16) & 123 & (12)\\
MWEvoViol & 62 & (6) & 57 & (6) & 83 & (8) & 262 & (26) & 256 & (25) & 200 & (20) & 97 & (10)\\
MWEvoNurture & 52 & (5) & 39 & (4) & 64 & (6) & 231 & (23) & 253 & (25) & 240 & (24) & 138 & (14)\\
MWEvoTraits & 104 & (10) & 134 & (13) & 180 & (18) & 271 & (27) & 163 & (16) & 104 & (10) & 61 & (6)\\
MWEvoDiff2 & 86 & (8) & 77 & (8) & 97 & (10) & 282 & (28) & 243 & (24) & 141 & (14) & 91 & (9)\\
		
			
	\hline
	\multicolumn{15}{c}{7. Men and women naturally differ}\\
	\nopagebreak
	\hline
	\nopagebreak
	MWNatStrong & 100 & (10) & 79 & (8) & 126 & (12) & 192 & (19) & 205 & (20) & 178 & (18) & 137 & (13)\\
MWNatAggress & 48 & (5) & 54 & (5) & 85 & (8) & 188 & (18) & 279 & (27) & 218 & (21) & 145 & (14)\\
MWNatCaring & 28 & (3) & 39 & (4) & 67 & (7) & 200 & (20) & 288 & (28) & 221 & (22) & 174 & (17)\\
MWNatTraits & 50 & (5) & 80 & (8) & 144 & (14) & 221 & (22) & 254 & (25) & 170 & (17) & 98 & (10)\\
MWNatDiff & 41 & (4) & 47 & (5) & 75 & (7) & 171 & (17) & 270 & (27) & 223 & (22) & 190 & (19)\\
		

	\pagebreak
	\hline
	Item name & \multicolumn{2}{c}{\rotatebox[origin=c]{70}{Strongly disagree}} &
	\multicolumn{2}{c}{\rotatebox[origin=c]{70}{Disagree}} &
	\multicolumn{2}{c}{\rotatebox[origin=c]{70}{Somewhat disagree}} &
	\multicolumn{2}{c}{\rotatebox[origin=c]{70}{Neither agree nor disagree} } &
	\multicolumn{2}{c}{\rotatebox[origin=c]{70}{Somewhat agree}} &
	\multicolumn{2}{c}{\rotatebox[origin=c]{70}{Agree}} &
	\multicolumn{2}{c}{\rotatebox[origin=c]{70}{Strongly agree}}  \\
	\hline
	\multicolumn{15}{c}{8. Men and women are equal}\\
	\hline
	\nopagebreak
	MWEqu & 40 & (4) & 35 & (3) & 100 & (10) & 115 & (11) & 180 & (18) & 243 & (24) & 304 & (30)\\
MWEquDiff & 108 & (11) & 111 & (11) & 161 & (16) & 199 & (20) & 188 & (18) & 129 & (13) & 121 & (12)\\
MWEquCulture & 32 & (3) & 40 & (4) & 74 & (7) & 211 & (21) & 228 & (22) & 214 & (21) & 218 & (21)\\
MWEquNoBio & 32 & (3) & 36 & (4) & 75 & (7) & 123 & (12) & 151 & (15) & 258 & (25) & 342 & (34)\\
MWEquInvent & 206 & (20) & 170 & (17) & 161 & (16) & 229 & (23) & 116 & (11) & 82 & (8) & 53 & (5)\\
			
			
			

		
\end{longtable}

\end{document}
